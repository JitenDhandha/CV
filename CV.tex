\documentclass{article}
\usepackage[utf8]{inputenc}
\usepackage[margin=1.5cm]{geometry}
\usepackage{natbib}
\usepackage{amssymb, amsmath}
\usepackage{xcolor}

\usepackage[colorlinks=True,urlcolor=blue]{hyperref}
\usepackage{array} % for defining new column types
\usepackage{fontawesome} % for icons
\usepackage{comment} % for commenting out sections
\usepackage{titlesec} % for custom section titles

% Define new column types for the tables
\newcolumntype{L}{>{\raggedleft}p{0.175\textwidth}}
\newcolumntype{R}{p{0.765\textwidth}}

% Custom section titles
\titleformat{\section}{\Large\bfseries}{\thesection}{}{}[{\titlerule[0.5pt]}]
\titleformat{\subsection}{\normalsize\bfseries}{\thesection}{}{}

\begin{document}

\small % Reduce font size for the whole document

%\begin{flushright}
%  \footnotesize
%  \textbf{Version v2.3}
%\end{flushright}

\begin{center}
\huge
\textbf{Jiten Dhandha} \\
\normalsize
PhD student - University of Cambridge \\
\end{center}

\begin{minipage}[ht]{0.6\linewidth}
	\faEnvelope~~\textbf{Email}: jvd29@cam.ac.uk / jitendhandha@gmail.com\\
	\faGithub~~\textbf{Github:} \href{https://github.com/JitenDhandha}{github.com/JitenDhandha}\\
	\faGlobe~~\textbf{Website:} \href{https://jitendhandha.com}{jitendhandha.com} \\
  \faPhone~~\textbf{Mobile:} +44(0)7442793684 \\
\end{minipage}
\begin{minipage}[ht]{0.34\linewidth}
	\begin{flushright}
    \textbf{ORCID:} \href{https://orcid.org/0000-0002-1481-0907}{0000-0002-1481-0907} \\
    \textbf{arXiv:} \href{https://arxiv.org/a/dhandha_j_1.html}{dhandha\_j\_1} \\
    \textbf{Google Scholar:} \href{https://scholar.google.com/citations?user=RjlmcA0AAAAJ}{Jiten Dhandha} \\
    \textbf{NASA/ADS:} \href{https://ui.adsabs.harvard.edu/search/?q=author%3A%22Dhandha%2C+Jiten%22}{Jiten Dhandha} \\
	\end{flushright}
\end{minipage}

\section*{Employment}

\begin{tabular}{L!{\vrule}R}
    Jul. 2024 - Aug. 2024 & \textbf{Summer volunteer internship}, Boustany Foundation, Monaco. Partnering with Open Cultural Center, a humanitarian NGO focused on providing teaching, advice and extracurricular activities to asylum-seekers and refugees in the Nea Kavala camp in Greece.  \\
    Jun. 2021 - Aug. 2021 & \textbf{Summer research project}, University of Manchester. Modelling the cosmological 21-cm signal in \href{https://bitbucket.org/Jacetoto/recfast-.vx/src/Recfast_JD_21cm_modelling/}{\texttt{Recfast++}} and \texttt{CosmoTherm} to study their synergy with CMB spectral distortions. Supervised by Prof. Jens Chluba. \\
    Jun. 2020 - Sep. 2020 & \textbf{Summer research project}, University of Manchester. Testing and debugging \texttt{LOFAR-VLBI} calibration/imaging pipeline for gravitational lenses. Supervised by Dr. Neal Jackson.\\
    Jul. 2019 - Sep. 2019 & \textbf{Summer Intern Programme}, British Petroleum / University of Manchester. Simulating mitigation techniques for sulphate reducing bacteria responsible for fouling crude oil. Supervised by Dr. Thomas Waigh. \\
\end{tabular}

\section*{Education}
\begin{tabular}{L!{\vrule}R}
	2022 - \textit{present} & \textbf{PhD in Astronomy}, Institute of Astronomy, University of Cambridge. Funded by \href{https://boustany-foundation.org/scholarship-programmes/astronomy-phd-cambridge/}{Boustany Astronomy Scholarship} \& \href{https://www.cambridgetrust.org/}{Isaac Newton Studentship} at Pembroke College. Supervised by Prof. Anastasia Fialkov and Dr. Eloy de Lera Acedo. \\
	2018 - 2022 & \textbf{MPhys. Physics with Astrophysics} First Class, University of Manchester. Project involved simulating turbulent molecular clouds in ISM and studying filament and star formation. Performed with Zoe Faes and supervised by Dr. Rowan Smith. \\
	2016 - 2018 & \textbf{All India Senior School Certificate Examination}, DPS - Modern Indian School, Doha, Qatar. Average of 95.2\% in AISSCE (A-level equivalent) examination.
\end{tabular}

\section*{Publications}

\subsection*{First Author}

\begin{tabular}{L!{\vrule}R}
% Start of first author papers 
    April 2024 &
    \textbf{J. Dhandha}, Z. Faes, R. J. Smith
    \href{https://ui.adsabs.harvard.edu/abs/2024MNRAS.529.4699D}{\textit{Decaying turbulence in molecular clouds: how does it affect filament networks and star formation?}},
    Monthly Notices of the Royal Astronomical Society, 529, 4699-4718 \\
% End of first author papers
\end{tabular}

\subsection*{Contributing Author}

\begin{tabular}{L!{\vrule}R}
% Start of contributing author papers 
    February 2024 &
    O. S. D. O'Hara, F. Dulwich, E. de Lera Acedo, \textbf{J. Dhandha}, T. Gessey-Jones, D. Anstey, A. Fialkov
    \href{https://ui.adsabs.harvard.edu/abs/2024arXiv240204008O}{\textit{Understanding spectral artefacts in SKA-LOW 21-cm cosmology experiments: the impact of cable reflections}},
    arXiv e-prints, arXiv:2402.04008 \\

    November 2023 &
    A. Fialkov, T. Gessey-Jones, \textbf{J. Dhandha}
    \href{https://ui.adsabs.harvard.edu/abs/2023arXiv231105366F}{\textit{Cosmic mysteries and the hydrogen 21-cm line: bridging the gap with lunar observations}},
    arXiv e-prints, arXiv:2311.05366 \\

    December 2022 &
    S. K. Acharya, \textbf{J. Dhandha}, J. Chluba
    \href{https://ui.adsabs.harvard.edu/abs/2022MNRAS.517.2454A}{\textit{Can accreting primordial black holes explain the excess radio background?}},
    Monthly Notices of the Royal Astronomical Society, 517, 2454-2461 \\

    February 2022 &
    S. Badole, D. Venkattu, N. Jackson, S. Wallace, \textbf{J. Dhandha}, P. Hartley, C. Riddell-Rovira, A. Townsend, L. K. Morabito, J. P. McKean
    \href{https://ui.adsabs.harvard.edu/abs/2022A&A...658A...7B}{\textit{High-resolution imaging with the International LOFAR Telescope: Observations of the gravitational lenses MG 0751+2716 and CLASS B1600+434}},
    Astronomy \& Astrophysics, 658, A7 \\
% End of contributing author papers
\end{tabular}

\section*{Talks}
\subsection*{Conference and Workshop talks}
\begin{tabular}{L!{\vrule}R}

    October 2024 & \textit{Constraining star-formation efficiency in the early Universe using JWST and the cosmic 21-cm signal}, Introduction to KICC, Kavli Institute for Cosmology, Cambridge. \\

    September 2024 & \textit{Constraining star-formation efficiency in the early Universe using JWST and the cosmic 21-cm signal}, 7th Global 21-cm Workshop, Raman Research Institute. \\

    May 2024 & \textit{Synergies between 21-cm experiments and JWST observations}, Reionization in Relic Radiation (R3), Institut d'Astrophysique Spatiale, Université Paris-Saclay. \\

    February 2024 & \textit{Synergies between 21-cm experiments and JWST observations}, Science with the 21-cm line, KICC, University of Cambridge. \\

    September 2023 & \textit{Bringing 21-cm simulations to the JWST era}, 2nd REACH Annual Meeting, University of Malta. \\

    September 2023 & \textit{FIlamEntary STructure Analysis (fiesta)}, AREPO-ISM workshop, University of Manchester. \\

    October 2022 & \textit{Can accreting primordial black holes explain the excess radio background?}, PDAT Laboratory, K. N. Toosi University of Technology (virtual webinar). \\

\end{tabular}

\subsection*{Outreach talks}
\begin{tabular}{L!{\vrule}R}

  October 2022 & \textit{Like beads on a string\ldots Where do massive stars in our Universe come from? A brief look into studying our cosmos}, Pembroke Papers, Pembroke College, University of Cambridge. \\

\end{tabular}

\section*{Grants and awards}

\begin{tabular}{L!{\vrule}R}

    April 2024 & \textbf{DiRAC Resource Allocation Committee 16th Call}, awarded 4.15M CPUh (worth £41,500) on DiRAC's COSMA-8 supercomputer. \\
    July 2022 & \textbf{Tessella Prize for Software} (£125), for outstanding work implementing software in Mphys project. \\
    April 2019 & \textbf{BP Achievement Award} (£1000), for best essay on petrophysical logging tools. \\
    December 2018 & \textbf{Physics Success Scholarship} (£2000), for academic excellence in physics and maths.\\

\end{tabular}

\begin{comment}
\section*{Posters Presentations}

\begin{tabular}{L!{\vrule}R}
	March 2021 & \href{https://github.com/htjb/Talks/blob/master/Posters/SKA_globalemu_March_2021/globalemu.pdf}{\textit{globalemu: A novel and robust approach for emulating the sky averaged 21-cm signal from the cosmic dawn and epoch of reionization}}, A Precursor View of the SKA Sky, Virtual Conference\\
	December 2019 & \href{https://drive.google.com/file/d/1dvgumyu4cXxXqoYxikU3DKOa4u_gpGzn/view}{\textit{REACH: Radio Experiment for the Analysis of Cosmic Hydrogen}}, Science At Low Frequencies VI, Arizona State University \\
	November 2019 & \href{https://github.com/htjb/Talks/blob/master/Posters/Cav_Graduate_Conf_REACH_Nov_2019/REACH_poster.pdf}{\textit{REACH: Radio Experiment for the Analysis of Cosmic Hydrogen}}, Cavendish Graduate Conference, University of Cambridge
\end{tabular}
\end{comment}

\section*{Conference organisation}

\begin{tabular}{L!{\vrule}R}
  February 2024  &\href{https://www.kicc.cam.ac.uk/events/kavli-science-themed-meetings/science-21-cm-hydrogen-line}{Kavli Science Focus: Science with the 21-cm line}, member of Organising Committee and session chair, KICC, University of Cambrdige. \\
\end{tabular}
\section*{Teaching responsibilities}

\begin{tabular}{L!{\vrule}R}

	Oct. 2024 - \textit{present} & \textbf{Co-Supervision: Kyle Wong} (Masters student) with Prof. Anastasia Fialkov.  Working on studying the impact of structure formation and cosmology on the hydrogen 21-cm signal from Cosmic Dawn. \\

	Oct. 2024 - \textit{present} & \textbf{Co-Supervision: Jacques Valkenberg} (Masters student) with Prof. Anastasia Fialkov and Dr. Sandro Tacchella. Working on studying the impact of metal enrichment and stellar IMF on the 21-cm signal of neutral hydrogen. \\

	Oct. 2023 - \textit{Jul. 2024} & \textbf{Co-Supervision: Rachel Incley} (Masters student) with Prof. Anastasia Fialkov. Working on comparison of Epoch of Reionization in simulation codes \texttt{21cmSPACE} and \texttt{C2-Ray}. \\

  Feb. 2023 - Mar. 2023 & \textbf{Demonstration of Part IA Scientific Computing} for 22 hours, University of Cambridge.\\

\end{tabular}

\section*{Software}

\begin{tabular}{L!{\vrule}R}
  \href{https://github.com/JitenDhandha/21cmExperiments}{\texttt{21cmExperiments}} & Maintainer: Comprehensive public repository/Google sheet of past \& ongoing 21-cm experiments. \\
	\href{https://github.com/JitenDhandha/CFit}{\texttt{CFit}} & Main author and maintainer: Smart curve fitting tool using method of least squares in Python.\\
	\href{https://fiesta-astro.readthedocs.io}{\texttt{fiesta}} & Main author and maintainer: Toolkit for analyzing filament networks and density field meshes. \\
	\href{https://github.com/JitenDhandha/luminobs}{\texttt{luminobs}} & Main author and maintainer: Compendium of high-redshift galaxy UVLF observations.\\
\end{tabular}

\section*{In the media}

\begin{tabular}{L!{\vrule}R}
	August 2021 & \href{https://www.astron.nl/most-detailed-ever-images-of-galaxies-revealed-using-lofar }{Most detailed-ever images of galaxies revealed using LOFAR}. Press release for LOFAR observations from ASTRON. \\
	August 2021 & \href{https://www.bbc.co.uk/news/science-environment-57998940}{Astronomers develop novel way to `see' first stars through fog of early Universe}. Press release for LOFAR observations from BBC. \\
\end{tabular}

\section*{Extracurricular activities}
\begin{tabular}{L!{\vrule}R}

  May 2023 - \textit{present} & \textbf{Inclusion and Fairness} committee member, Institute of Astronomy, University of Cambridge. \\

  Jul. 2023 - Jul. 2024 & \textbf{Graduate Parlour}, Ethnic Minorities officer, Pembroke College, University of Cambridge. \\

  Oct. 2022 - \textit{present} & \textbf{Postgraduate Forum} representative, Institute of Astronomy, University of Cambridge. \\

  Oct. 2022 - Apr. 2023 & \textbf{Pembroke Papers} committee memeber, Pembroke College, University of Cambridge. \\

  Sep. 2021 - Jul. 2022 & \textbf{Student Representative} representing astronomy/astrophysics, University of Manchester. \\

  Jul. 2020 - Jul. 2022 & \textbf{Touch Rugby Society}, Inclusion officer and COVID-19 safety officer, University of Manchester. \\

	Sep. 2019 - Jun. 2020 & \textbf{Peer-Assisted Study Session} leader, Peer Support Scheme, University of Manchester. \\

	Nov. 2016 - \textit{present} & \textbf{English Wikipedia}, volunteer editor. \\
\end{tabular}

\section*{Skills}

\begin{tabular}{L!{\vrule}R}
	Programming & \textbf{Proficient}: Python, MATLAB, \textbf{Experienced}: C++, Java\\
	Markup & \textbf{Experienced}: LaTeX, Wikitext, \textbf{Intermediate}:  HTML, CSS, reStructuredText, Markdown \\
	Languages & \textbf{Proficient}: English, Hindi, \textbf{Intermediate}: Gujarati
\end{tabular}

\end{document}