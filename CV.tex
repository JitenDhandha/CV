\documentclass{article}
\usepackage[utf8]{inputenc}
\usepackage[margin=1.5cm]{geometry}
\usepackage{natbib}
\usepackage{amssymb, amsmath}
\usepackage{xcolor}

\usepackage[colorlinks=True,urlcolor=blue]{hyperref}
\usepackage{array} % for defining new column types
\usepackage{fontawesome} % for icons
\usepackage{comment} % for commenting out sections
\usepackage{titlesec} % for custom section titles
\usepackage{longtable} % for long tables that span multiple pages

% Define new column types for the tables
\newcolumntype{L}{>{\raggedleft}p{0.175\textwidth}}
\newcolumntype{R}{p{0.765\textwidth}}

% Custom section titles
\titleformat{\section}{\Large\bfseries}{\thesection}{}{}[{\titlerule[0.5pt]}]
\titleformat{\subsection}{\normalsize\bfseries}{\thesection}{}{}

\begin{document}

\small % Reduce font size for the whole document

%\begin{flushright}
%  \footnotesize
%  \textbf{Version v2.3}
%\end{flushright}

\begin{center}
\huge
\textbf{Jiten Dhandha} \\
\normalsize
PhD student - University of Cambridge \\
\end{center}

\begin{minipage}[ht]{0.6\linewidth}
	\faEnvelope~~\textbf{Email}: \href{mailto:jvd29@cam.ac.uk}{jvd29@cam.ac.uk} / \href{mailto:jitendhandha@gmail.com}{jitendhandha@gmail.com} \\
	\faGithub~~\textbf{Github:} \href{https://github.com/JitenDhandha}{github.com/JitenDhandha}\\
	\faGlobe~~\textbf{Website:} \href{https://jitendhandha.com}{jitendhandha.com} \\
  \faPhone~~\textbf{Mobile:} +44(0)7442793684 \\
\end{minipage}
\begin{minipage}[ht]{0.34\linewidth}
	\begin{flushright}
    \textbf{ORCID:} \href{https://orcid.org/0000-0002-1481-0907}{0000-0002-1481-0907} \\
    \textbf{arXiv:} \href{https://arxiv.org/a/dhandha_j_1.html}{dhandha\_j\_1} \\
    \textbf{Google Scholar:} \href{https://scholar.google.com/citations?user=RjlmcA0AAAAJ}{Jiten Dhandha} \\
    \textbf{NASA/ADS:} \href{https://ui.adsabs.harvard.edu/search/?q=author%3A%22Dhandha%2C+Jiten%22}{Jiten Dhandha} \\
	\end{flushright}
\end{minipage}

\section*{Research interests}

My research focuses on the study of the early Universe, spanning the \textbf{Dark Ages, Cosmic Dawn, and the Epoch of Reionization} (redshift $z\sim 6 - 50$). My expertise lies in the theoretical modelling of the \textbf{cosmic 21-cm signal} of neutral hydrogen, and of \textbf{stellar and galactic astrophysics}; this includes numerical simulations of present-day ($z \sim 0$) Population~I star-formation in molecular clouds, as well as analytic and semi-numerical simulations of high-redshift ($z \gtrsim 6$) Population~II and Population~III stars. I have extensively used \textbf{machine learning and Bayesian inference techniques} to accelerate simulations and perform parameter inference, combining data from multi-wavelength facilities (e.g. from JWST and 21-cm experiments).

\begin{comment}
\section*{Skills}

\begin{longtable}{L!{\vrule}R}
	Programming & \textbf{Proficient}: Python, MATLAB, \textbf{Experienced}: C++, Java\\
	Markup & \textbf{Experienced}: LaTeX, Wikitext, \textbf{Intermediate}:  HTML, CSS, reStructuredText, Markdown \\
	Languages & \textbf{Proficient}: English, Hindi, \textbf{Intermediate}: Gujarati
\end{longtable}
\end{comment}

\section*{Education}
\begin{longtable}{L!{\vrule}R}
	2022 - \textit{present} & \textbf{PhD in Astronomy}, Institute of Astronomy, University of Cambridge. Funded by \href{https://boustany-foundation.org/scholarship-programmes/astronomy-phd-cambridge/}{Boustany Astronomy Scholarship} \& \href{https://www.cambridgetrust.org/partners/institute-of-astronomy}{Isaac Newton Studentship} at Pembroke College. Supervised by Prof. Anastasia Fialkov and Dr. Eloy de Lera Acedo. \\
	2018 - 2022 & \textbf{MPhys. Physics with Astrophysics} First Class, University of Manchester. Project involved simulating turbulent molecular clouds in ISM and studying filament and star formation. Performed with Zoe Faes and supervised by Dr. Rowan Smith. \\
	2016 - 2018 & \textbf{All India Senior School Certificate Examination}, DPS - Modern Indian School, Doha, Qatar. Average of 95.2\% in AISSCE examination.
\end{longtable}

\section*{Employment}

\begin{longtable}{L!{\vrule}R}
    Jul. 2024 - Aug. 2024 & \textbf{Summer volunteer internship}, Boustany Foundation, Monaco. Partnering with Open Cultural Center, a humanitarian NGO focused on providing teaching, advice and extracurricular activities to asylum-seekers and refugees in the Nea Kavala camp in Greece.  \\
    Jun. 2021 - Aug. 2021 & \textbf{Summer research project}, University of Manchester. Modelling the cosmological 21-cm signal in \href{https://bitbucket.org/Jacetoto/recfast-.vx/src/Recfast_JD_21cm_modelling/}{\texttt{Recfast++}} and \texttt{CosmoTherm} to study their synergy with CMB spectral distortions. Supervised by Prof. Jens Chluba. \\
    Jun. 2020 - Sep. 2020 & \textbf{Summer research project}, University of Manchester. Testing and debugging \texttt{LOFAR-VLBI} calibration/imaging pipeline for gravitational lenses. Supervised by Dr. Neal Jackson.\\
    Jul. 2019 - Sep. 2019 & \textbf{Summer intern programme}, British Petroleum / University of Manchester. Simulating mitigation techniques for sulphate reducing bacteria responsible for fouling crude oil. Supervised by Dr. Thomas Waigh. \\
\end{longtable}

\section*{Grants and awards}

\begin{longtable}{L!{\vrule}R}

    April 2024 & \textbf{DiRAC Resource Allocation Committee 16th Call}, awarded 4.15M CPUh (worth £41,500) on DiRAC's COSMA-8 supercomputer. \\
    July 2022 & \textbf{Tessella Prize for Software} (£125), for outstanding work implementing software in Mphys project. \\
    March 2022 & \textbf{Boustany Scholarship for Astronomy} ($\sim$ £100,530), for PhD at University of Cambridge. \\
    April 2019 & \textbf{BP Achievement Award} (£1000), for best essay on petrophysical logging tools. \\
    December 2018 & \textbf{Physics Success Scholarship} (£2000), for academic excellence in physics and maths.\\

\end{longtable}

\clearpage

\section*{Services}

\subsection*{Conference and Workshop organization}

\begin{longtable}{L!{\vrule}R}

  July 2025  & One day workshop: Radio cosmology and science with the 21-cm signal, member of Organising Committee and session chair, KICC, University of Cambridge. \\

  February 2024 & \href{https://www.kicc.cam.ac.uk/events/kavli-science-themed-meetings/science-21-cm-hydrogen-line}{Kavli Science Focus: Science with the 21-cm line}, member of Organising Committee and session chair, KICC, University of Cambridge. \\
\end{longtable}

\subsection*{Other committees}

\begin{longtable}{L!{\vrule}R}

  Feb. 2025 - Mar. 2025 & \textbf{International Womens Day Committee} member, Institute of Astronomy, UoC. \\

  May 2023 - \textit{present} & \textbf{EDI Inclusion and Fairness} subgroup member, Institute of Astronomy, UoC. \\

  Jul. 2023 - Jul. 2024 & \textbf{Graduate Parlour}, Ethnic Minorities officer, Pembroke College, UoC. \\

  Oct. 2022 - \textit{present} & \textbf{Postgraduate Forum} representative, Institute of Astronomy, UoC. \\

  Oct. 2022 - Apr. 2023 & \textbf{Pembroke Papers} committee memeber, Pembroke College, UoC. \\
\end{longtable}

\section*{Teaching responsibilities}

\begin{longtable}{L!{\vrule}R}

	Oct. 2025 - \textit{present} & \textbf{Co-Supervision: Saughn Sekhon} (Part~III student) with Prof. Anastasia Fialkov and Dr. Peter Sims. Work on machine learning accelerated Bayesian inference of early Universe star-formation. \\

	Oct. 2025 - \textit{present} & \textbf{Co-Supervision: Seowon Jung} (Part~III student) with Prof. Anastasia Fialkov and Prof. Alexander Belyaev. Work on probing the nature of dark matter with 21-cm cosmology. \\

	Oct. 2024 - Jul. 2025 & \textbf{Co-Supervision: Jacques Valkenberg} (MPhil. student) with Prof. Anastasia Fialkov and Prof. Sandro Tacchella. Work on studying the impact of metal enrichment and Population~II stellar IMF on the 21-cm signal of neutral hydrogen. \\

	Oct. 2024 - Jul. 2025 & \textbf{Co-Supervision: Kyle Wong} (Part~III student) with Prof. Anastasia Fialkov. Work on studying the impact of varying cosmology and matter power spectrum on the 21-cm signal of neutral hydrogen. \\

	Oct. 2023 - Jul. 2024 & \textbf{Co-Supervision: Jamie Incley} (Part~III student) with Prof. Anastasia Fialkov. Work on comparison of Epoch of Reionization in simulation codes \texttt{21cmSPACE} and \texttt{C2-Ray}. \\

  Feb. 2023 - Mar. 2023 & \textbf{Demonstration of Part IA Scientific Computing} for 22 hours, University of Cambridge.\\

\end{longtable}

\section*{Talks}

\subsection*{Invited talks}

\begin{longtable}{L!{\vrule}R}

    September 2025 & \textit{An overview of 21-cm cosmology and how we can constrain the discovery space of the 21-cm signal}, 8th Global 21-cm Workshop, California Institute of Technology. \\
\end{longtable}


\subsection*{Conference and Workshop talks}
\begin{longtable}{L!{\vrule}R}

    June 2025 & \textit{Exploiting synergies between JWST and cosmic 21-cm observations to uncover star formation in the early Universe}, SKAO General Science Meeting 2025, Görlitz, Germany. \\

    October 2024 & \textit{Constraining star-formation efficiency in the early Universe using JWST and the cosmic 21-cm signal}, Introduction to KICC, Kavli Institute for Cosmology, Cambridge. \\

    September 2024 & \textit{Constraining star-formation efficiency in the early Universe using JWST and the cosmic 21-cm signal}, 7th Global 21-cm Workshop, Raman Research Institute. \\

    May 2024 & \textit{Synergies between 21-cm experiments and JWST observations}, Reionization in Relic Radiation (R3), Institut d'Astrophysique Spatiale, Université Paris-Saclay. \\

    February 2024 & \textit{Synergies between 21-cm experiments and JWST observations}, Science with the 21-cm line, KICC, University of Cambridge. \\

    September 2023 & \textit{FIlamEntary STructure Analysis (fiesta)}, AREPO-ISM workshop, University of Manchester. \\

    October 2022 & \textit{Can accreting primordial black holes explain the excess radio background?}, PDAT Laboratory, K. N. Toosi University of Technology (virtual webinar). \\

\end{longtable}


\subsection*{Meeting talks}
\begin{longtable}{L!{\vrule}R}

    October 2025 & \textit{An overview of 21-cm cosmology and how we can constrain the discovery space of the 21-cm signal}, mini-workshop on 21-cm signal (headed by Prof. Naoki Yoshida), Department of Physics, University of Tokyo. \\

    September 2025 & \textit{What do we know about the 21-cm signal so far?}, group meeting presentation (Gravitational Wave Astrophysics Group headed by Prof. Michela Mapelli), Centre for Astronomy of Heidelberg University. \\

    September 2023 & \textit{Bringing 21-cm simulations to the JWST era}, 2nd REACH Annual Meeting, University of Malta. \\

\end{longtable}

\subsection*{Outreach talks}
\begin{longtable}{L!{\vrule}R}

  October 2022 & \textit{Like beads on a string\ldots Where do massive stars in our Universe come from? A brief look into studying our cosmos}, Pembroke Papers, Pembroke College, University of Cambridge. \\

\end{longtable}

\begin{comment}
\section*{Posters Presentations}

\begin{longtable}{L!{\vrule}R}
	March 2021 & \href{https://github.com/htjb/Talks/blob/master/Posters/SKA_globalemu_March_2021/globalemu.pdf}{\textit{globalemu: A novel and robust approach for emulating the sky averaged 21-cm signal from the cosmic dawn and epoch of reionization}}, A Precursor View of the SKA Sky, Virtual Conference\\
	December 2019 & \href{https://drive.google.com/file/d/1dvgumyu4cXxXqoYxikU3DKOa4u_gpGzn/view}{\textit{REACH: Radio Experiment for the Analysis of Cosmic Hydrogen}}, Science At Low Frequencies VI, Arizona State University \\
	November 2019 & \href{https://github.com/htjb/Talks/blob/master/Posters/Cav_Graduate_Conf_REACH_Nov_2019/REACH_poster.pdf}{\textit{REACH: Radio Experiment for the Analysis of Cosmic Hydrogen}}, Cavendish Graduate Conference, University of Cambridge
\end{longtable}
\end{comment}

\section*{Workshops attended}
\begin{longtable}{L!{\vrule}R}
    September 2025 & \href{https://indico.physi.uni-heidelberg.de/event/1053/}{IWR School on Machine Learning for Fundamental Physics}, Faculty of Mathematics and Computer Science, Heidelberg University. \\

\end{longtable}

\section*{Software development}

\begin{longtable}{L!{\vrule}R}
  \href{https://github.com/JitenDhandha/eor_limits}{\texttt{eor\_limits}} & Author and maintainer of the new beta-release of the web-app: \href{https://eorlimits.streamlit.app/}{eorlimits.streamlit.app}, which allows users to compare upper limits on the 21-cm power spectrum from different experiments.\\
  \href{https://github.com/JitenDhandha/21cmSimulators}{\texttt{21cmSimulators}} & Main author and maintainer: Community-led and comprehensive public repository of the widely used 21-cm signal simulation codes. \\
  \href{https://github.com/JitenDhandha/21cmExperiments}{\texttt{21cmExperiments}} & Main author and maintainer: Community-led and comprehensive public repository of past \& ongoing 21-cm experiments. \\
	\href{https://github.com/JitenDhandha/CFit}{\texttt{CFit}} & Main author and maintainer: Smart curve fitting tool using method of least squares in Python.\\
	\href{https://fiesta-astro.readthedocs.io}{\texttt{fiesta}} & Main author and maintainer: Toolkit for analyzing filament networks and density field meshes. \\
\end{longtable}

\section*{In the media}

\begin{longtable}{L!{\vrule}R}
	August 2021 & \href{https://www.astron.nl/most-detailed-ever-images-of-galaxies-revealed-using-lofar }{Most detailed-ever images of galaxies revealed using LOFAR}. Press release for LOFAR observations from ASTRON. \\
	August 2021 & \href{https://www.bbc.co.uk/news/science-environment-57998940}{Astronomers develop novel way to `see' first stars through fog of early Universe}. Press release for LOFAR observations from BBC. \\
\end{longtable}

\section*{Extracurricular activities}
\begin{longtable}{L!{\vrule}R}

  Sep. 2021 - Jul. 2022 & \textbf{Student Representative} representing astronomy/astrophysics, UoM. \\

  Jul. 2020 - Jul. 2022 & \textbf{Touch Rugby Society}, Inclusion officer and COVID-19 safety officer, UoM. \\

	Sep. 2019 - Jun. 2020 & \textbf{Peer-Assisted Study Session} leader, Peer Support Scheme,UoM. \\

	Nov. 2016 - \textit{present} & \textbf{English Wikipedia}, volunteer editor. \\
\end{longtable}

\newpage

\section*{Publications}

% Start of paper counter 
I have \textbf{3 first author} publications and \textbf{9 contributing author} publications.
% End of paper counter

\subsection*{First Author}

\begin{longtable}{L!{\vrule}R}
% Start of first author papers 
    October 2025 &
    \textbf{J. Dhandha}, A. Fialkov, T. Gessey-Jones, H. T. J. Bevins, S. Tacchella, S. Pochinda, E. de Lera Acedo, S. Singh, R. Barkana
    \href{https://ui.adsabs.harvard.edu/abs/2025arXiv250813761D}{\textit{Narrowing the discovery space of the cosmological 21-cm signal using multi-wavelength constraints}},
    Accepted in Monthly Notices of the Royal Astronomical Society,  \\

    September 2025 &
    \textbf{J. Dhandha}, A. Fialkov, T. Gessey-Jones, H. T. J. Bevins, S. Tacchella, S. Pochinda, E. de Lera Acedo, S. Singh, R. Barkana
    \href{https://ui.adsabs.harvard.edu/abs/2025MNRAS.542.2292D}{\textit{Exploiting synergies between JWST and cosmic 21-cm observations to uncover star formation in the early Universe}},
    Monthly Notices of the Royal Astronomical Society, 542, 2292-2322 \\

    April 2024 &
    \textbf{J. Dhandha}, Z. Faes, R. J. Smith
    \href{https://ui.adsabs.harvard.edu/abs/2024MNRAS.529.4699D}{\textit{Decaying turbulence in molecular clouds: how does it affect filament networks and star formation?}},
    Monthly Notices of the Royal Astronomical Society, 529, 4699-4718 \\
% End of first author papers
\end{longtable}

\subsection*{Contributing Author}

\begin{longtable}{L!{\vrule}R}
% Start of contributing author papers 
    October 2025 &
    S. Munshi, F. G. Mertens, J. K. Chege, L. V. E. Koopmans, A. R. Offringa, B. Semelin, R. Barkana, \textbf{J. Dhandha}, A. Fialkov, R. M\'eriot, S. Sikder, A. Bracco, S. A. Brackenhoff, E. Ceccotti, R. Ghara, S. Ghosh, I. Hothi, M. Mevius, P. Ocvirk, A. K. Shaw, S. Yatawatta, P. Zarka
    \href{https://ui.adsabs.harvard.edu/abs/2025MNRAS.542.2785M}{\textit{Improved upper limits on the 21-cm signal power spectrum at z = 17.0 and z = 20.3 from an optimal field observed with NenuFAR}},
    Monthly Notices of the Royal Astronomical Society, 542, 2785-2807 \\

    August 2025 &
    B. Liu, D. Kessler, T. Gessey-Jones, \textbf{J. Dhandha}, A. Fialkov, Y. Sibony, G. Meynet, V. Bromm, R. Barkana
    \href{https://ui.adsabs.harvard.edu/abs/2025MNRAS.541.3113L}{\textit{Effects of chemically homogeneous evolution of the first stars on the 21-cm signal and reionization}},
    Monthly Notices of the Royal Astronomical Society, 541, 3113-3133 \\

    July 2025 &
    J. Wasserman, E. Zackrisson, \textbf{J. Dhandha}, A. Fialkov, L. Noble, S. Majumdar
    \href{https://ui.adsabs.harvard.edu/abs/2025arXiv250721764W}{\textit{Ultraviolet photon production rates of the first stars: Impact on the He II $\lambda 1640 \r{A}$ emission line from primordial star clusters and the 21-cm signal from cosmic dawn}},
    arXiv e-prints, arXiv:2507.21764 \\

    March 2025 &
    O. S. D. O'Hara, Q. Gueuning, E. de Lera Acedo, F. Dulwich, J. Cumner, D. Anstey, A. Brown, A. Fialkov, \textbf{J. Dhandha}, A. Faulkner, Y. Liu
    \href{https://ui.adsabs.harvard.edu/abs/2025MNRAS.538...31O}{\textit{Uncovering the effects of array mutual coupling in 21-cm experiments with the SKA-Low radio telescope}},
    Monthly Notices of the Royal Astronomical Society, 538, 31-48 \\

    February 2025 &
    S. Pochinda, \textbf{J. Dhandha}, A. Fialkov, E. de Lera Acedo
    \href{https://ui.adsabs.harvard.edu/abs/2025arXiv250200852P}{\textit{Cosmological super-resolution of the 21-cm signal}},
    Accepted at NeurIPS 2024 (Machine Learning and the Physical Sciences Workshop), arXiv:2502.00852 \\

    September 2024 &
    O. S. D. O'Hara, F. Dulwich, E. de Lera Acedo, \textbf{J. Dhandha}, T. Gessey-Jones, D. Anstey, A. Fialkov
    \href{https://ui.adsabs.harvard.edu/abs/2024MNRAS.533.2876O}{\textit{Understanding spectral artefacts in SKA-Low 21-cm cosmology experiments: the impact of cable reflections}},
    Monthly Notices of the Royal Astronomical Society, 533, 2876-2892 \\

    May 2024 &
    A. Fialkov, T. Gessey-Jones, \textbf{J. Dhandha}
    \href{https://ui.adsabs.harvard.edu/abs/2024RSPTA.38230068F}{\textit{Cosmic mysteries and the hydrogen 21-cm line: bridging the gap with lunar observations}},
    Philosophical Transactions of the Royal Society of London Series A, 382, 20230068 \\

    December 2022 &
    S. K. Acharya, \textbf{J. Dhandha}, J. Chluba
    \href{https://ui.adsabs.harvard.edu/abs/2022MNRAS.517.2454A}{\textit{Can accreting primordial black holes explain the excess radio background?}},
    Monthly Notices of the Royal Astronomical Society, 517, 2454-2461 \\

    February 2022 &
    S. Badole, D. Venkattu, N. Jackson, S. Wallace, \textbf{J. Dhandha}, P. Hartley, C. Riddell-Rovira, A. Townsend, L. K. Morabito, J. P. McKean
    \href{https://ui.adsabs.harvard.edu/abs/2022A&A...658A...7B}{\textit{High-resolution imaging with the International LOFAR Telescope: Observations of the gravitational lenses MG 0751+2716 and CLASS B1600+434}},
    Astronomy \& Astrophysics, 658, A7 \\
% End of contributing author papers
\end{longtable}

\end{document}