\documentclass{article}
\usepackage[utf8]{inputenc}
\usepackage[margin=1.5cm]{geometry}
\usepackage{natbib}
\usepackage{amssymb, amsmath}
\usepackage[colorlinks=True]{hyperref}
\hypersetup{urlcolor=blue}
\usepackage{array}
\usepackage{fontawesome}
\usepackage{comment}

% Define new column types for the tables
\newcolumntype{L}{>{\raggedleft}p{0.18\textwidth}}
\newcolumntype{R}{p{0.76\textwidth}}
\newcolumntype{l}{>{\raggedleft}p{0.25\textwidth}}
\newcolumntype{r}{p{0.71\textwidth}}

\begin{document}

\small % Reduce font size for the whole document

\begin{center}
\huge
\textbf{Jiten Dhandha} \\
\normalsize
PhD student - University of Cambridge
\end{center}

\begin{minipage}[ht]{0.6\linewidth}
	\faEnvelope~~\textbf{Email}: jvd29@cam.ac.uk / jitendhandha@gmail.com\\
	\faGithub~~\textbf{Github:} \href{https://github.com/JitenDhandha}{github.com/JitenDhandha}\\
	\faGlobe~~\textbf{Website:} \href{https://jitendhandha.com}{jitendhandha.com} \\
  \faPhone~~\textbf{Mobile}: +44(0)7729396746 \\
\end{minipage}
\begin{minipage}[ht]{0.3\linewidth}
	\begin{flushright}
	\end{flushright}
\end{minipage}

\section*{Employment}
\begin{tabular}{L!{\vrule}R}
    Jun. 2021 - Aug. 2021 & \textbf{Summer research project}, University of Manchester. Modelling the cosmological 21-cm signal in \href{https://bitbucket.org/Jacetoto/recfast-.vx/src/Recfast_JD_21cm_modelling/}{\texttt{Recfast++}} and \texttt{CosmoTherm} to study their synergy with CMB spectral distortions. Supervised by Prof. Jens Chluba. \\
    Jun. 2020 - Sep. 2020 & \textbf{Summer research project}, University of Manchester. Testing and debugging \texttt{LOFAR-VLBI} calibration/imaging pipeline for gravitational lenses. Supervised by Dr. Neal Jackson.\\
    Jul. 2019 - Sep. 2019 & \textbf{Summer Intern Programme}, British Petroleum / University of Manchester. Simulating mitigation techniques for sulphate reducing bacteria responsible for fouling crude oil. Supervised by Dr. Thomas Waigh. \\
\end{tabular}

\section*{Education}
\begin{tabular}{L!{\vrule}R}
	2022 - \textit{present} & \textbf{PhD in Astronomy}, Institute of Astronomy, University of Cambridge. Supervised by Prof. Anastasia Fialkov and Dr. Eloy de Lera Acedo. \\
	2018 - 2022 & \textbf{MPhys. Physics with Astrophysics, First Class}, University of Manchester. Project involved simulating turbulent molecular clouds in ISM and studying filament and star formation. Performed with Zoe Faes and supervised by Dr. Rowan Smith. \\
	2016 - 2018 & \textbf{All India Senior School Certificate Examination}, DPS - Modern Indian School, Doha, Qatar. Average of 95.2\% in AISSCE (A-level equivalent) examination.
\end{tabular}

\section*{Publications}

\subsubsection*{First Author}

\begin{tabular}{L!{\vrule}R}

    July 2023 & \textbf{J. Dhandha}, Z. Faes, R. J. Smith. \href{https://arxiv.org/abs/2307.12428}{\textit{Decaying turbulence in molecular clouds: how does it affect filament networks and star formation?}}, arXiv:astro-ph.GA, arXiv:2307.12428. \\

\end{tabular}

\subsubsection*{Contributing Author}

\begin{tabular}{L!{\vrule}R}

    March 2024 & A. Fialkov, T. Gessey-Jones, \textbf{J. Dhandha}. \href{https://doi.org/10.1098/rsta.2023.0068}{\textit{Cosmic mysteries and the hydrogen 21-cm line: bridging the gap with lunar observations}}, Philosophical Transactions of the Royal Society A, Volume 382, Issue 2271, arXiv:2311.05366. \\

    February 2024 & O. S. D. O'Hara, F. Dulwich, E. de Lera Acedo, \textbf{J. Dhandha}, T. Gessey-Jones, D. Anstey, A. Fialkov. \href{https://arxiv.org/abs/2402.04008}{\textit{Understanding spectral artefacts in SKA-LOW 21-cm cosmology experiments: the impact of cable reflections}}, arXiv:astro-ph.CO, arXiv:2402.04008. \\

	  September 2022 & S. K. Acharya, \textbf{J. Dhandha}, J. Chluba. \href{https://doi.org/10.1093/mnras/stac2739}{\textit{Can accreting primordial black holes explain the excess radio background?}}, Monthly Notices of the Royal Astronomy Societ, Volume 517, Issue 2, Pages 2454-2461, arXiv:2208.03816. \\

	  February 2022 & S. Badole, D. Venkattu, N. Jackson, S. Wallace, \textbf{J. Dhandha}, P. Hartley, C. Riddell-Rovira, A. Townsend, L. K. Morabito, J. P. McKean. \href{https://doi.org/10.1051/0004-6361/202141227}{\textit{High-resolution imaging with the International LOFAR Telescope: Observations of the gravitational lenses MG 0751+2716 and CLASS B1600+434}}, Astronomy \& Astrophysics, Volume 658, Issue 11, arXiv:2108.07293. \\

\end{tabular}

\section*{Talks}
\subsubsection*{Conference and Workshop talks}
\begin{tabular}{L!{\vrule}R}

    February 2024 & \textit{Synergies between 21-cm experiments and JWST observations}, Science with the 21-cm line, KICC, University of Cambridge. \\

    September 2023 & \textit{Bringing 21-cm simulations to the JWST era}, REACH Annual Meeting, University of Malta. \\

    September 2023 & \textit{FIlamEntary STructure Analysis (fiesta)}, AREPO-ISM workshop, University of Manchester. \\

    October 2022 & \textit{Can accreting primordial black holes explain the excess radio background?}, PDAT Laboratory, K. N. Toosi University of Technology (virtual webinar). \\

\end{tabular}

\subsubsection*{Outreach talks}
\begin{tabular}{L!{\vrule}R}

  October 2022 & \textit{Like beads on a string... Where do massive stars in our Universe come from? A brief look into studying our cosmos}, Pembroke Papers, Pembroke College, University of Cambridge. \\

\end{tabular}

\section*{Grants and awards}

\begin{tabular}{L!{\vrule}R}

    July 2022 & \textbf{Tessella Prize for Software} (£125), for outstanding work implementing software in Mphys project. \\
    April 2019 & \textbf{BP Achievement Award} (£1000), for best essay on petrophysical logging tools. \\
    December 2018 & \textbf{Physics Success Scholarship} (£2000), for academic excellence in physics and maths.\\

\end{tabular}

\begin{comment}
\section*{Posters Presentations}

\begin{tabular}{L!{\vrule}R}
	March 2021 & \href{https://github.com/htjb/Talks/blob/master/Posters/SKA_globalemu_March_2021/globalemu.pdf}{\textit{globalemu: A novel and robust approach for emulating the sky averaged 21-cm signal from the cosmic dawn and epoch of reionization}}, A Precursor View of the SKA Sky, Virtual Conference\\
	December 2019 & \href{https://drive.google.com/file/d/1dvgumyu4cXxXqoYxikU3DKOa4u_gpGzn/view}{\textit{REACH: Radio Experiment for the Analysis of Cosmic Hydrogen}}, Science At Low Frequencies VI, Arizona State University \\
	November 2019 & \href{https://github.com/htjb/Talks/blob/master/Posters/Cav_Graduate_Conf_REACH_Nov_2019/REACH_poster.pdf}{\textit{REACH: Radio Experiment for the Analysis of Cosmic Hydrogen}}, Cavendish Graduate Conference, University of Cambridge
\end{tabular}
\end{comment}

\section*{Conference organisation}

\begin{tabular}{L!{\vrule}R}
  February 2024  &\href{https://www.kicc.cam.ac.uk/events/kavli-science-themed-meetings/science-21-cm-hydrogen-line}{Kavli Science Focus: Science with the 21-cm line}, member of Organising Committee and session chair, KICC, University of Cambrdige. \\
\end{tabular}
\section*{Teaching responsibilities}

\begin{tabular}{L!{\vrule}R}

	Oct. 2023 - \textit{present} & \textbf{Co-Supervision} of Rachel Incley (Masters student) with Prof. Anastasia Fialkov. Working on comparison of Epoch of Reionization in simulation codes \texttt{21cmSPACE} and \texttt{C2-Ray}. \\
  Feb. 2023 - Mar. 2023 & \textbf{Demonstration of Part IA Scientific Computing} for 22 hours, University of Cambridge.\\

\end{tabular}

\section*{Software}

\begin{tabular}{L!{\vrule}R}
	\href{https://github.com/JitenDhandha/CFit}{\texttt{CFit}} & Main author and maintainer: Smart curve fitting tool using method of least squares in Python.\\
	\href{https://fiesta-astro.readthedocs.io}{\texttt{fiesta}} & Main author and maintainer: Toolkit for analyzing filament networks and density field meshes. \\
	\href{https://github.com/JitenDhandha/luminobs}{\texttt{luminobs}} & Main author and maintainer: Compendium of high-redshift galaxy UVLF observations.\\
\end{tabular}

\section*{In the media}

\begin{tabular}{L!{\vrule}R}
	August 2021 & \href{https://www.astron.nl/most-detailed-ever-images-of-galaxies-revealed-using-lofar }{Most detailed-ever images of galaxies revealed using LOFAR}. Press release for LOFAR observations from ASTRON. \\
	August 2021 & \href{https://www.bbc.co.uk/news/science-environment-57998940}{Astronomers develop novel way to ‘see’ first stars through fog of early Universe}. Press release for LOFAR observations from BBC. \\
\end{tabular}

\section*{Extracurricular activities}
\begin{tabular}{L!{\vrule}R}

  May 2023 - \textit{present} & \textbf{Inclusion and Fairness} committee member, Institute of Astronomy, University of Cambridge. \\

  Jul. 2023 - \textit{present} & \textbf{Graduate Parlour}, Ethnic Minorities officer, Pembroke College, University of Cambridge. \\

  Oct. 2022 - \textit{present} & \textbf{Postgraduate Forum} representative, Institute of Astronomy, University of Cambridge. \\

  Oct. 2022 - Apr. 2023 & \textbf{Pembroke Papers} committee memeber, Pembroke College, University of Cambridge. \\

  Sep. 2021 - Jul. 2022 & \textbf{Student Representative} representing astronomy/astrophysics, University of Manchester. \\

  Jul. 2020 - Jul. 2022 & \textbf{Touch Rugby Society}, Inclusion officer and COVID-19 safety officer, University of Manchester. \\

	Sep. 2019 - Jun. 2020 & \textbf{Peer-Assisted Study Session} leader, Peer Support Scheme, University of Manchester. \\

	Nov. 2016 - \textit{present} & \textbf{English Wikipedia}, volunteer editor. \\
\end{tabular}

\section*{Skills}

\begin{tabular}{L!{\vrule}R}
	Programming & \textbf{Proficient}: Python, MATLAB, \textbf{Experienced}: C++, Java\\
	Markup & \textbf{Experienced}: LaTeX, Wikitext, \textbf{Intermediate}:  HTML, CSS, reStructuredText, Markdown \\
	Languages & \textbf{Proficient}: English, Hindi, \textbf{Intermediate}: Gujarati
\end{tabular}

%\section*{References}

%\textbf{Dr Eloy de Lera Acedo}, PhD Supervisor, Head of Cavendish Radio Cosmology and Principle Investigator, Cavendish Astrophysics, University of Cambridge. \textbf{Email:}ed330@cam.ac.uk\\
%\\
%\textbf{Dr Will Handley}, PhD Supervisor, Royal Society University Research Fellow, Cavendish Astrophysics, University of Cambridge. \textbf{Email:} wh260@cam.ac.uk\\
%\\
%\textbf{Dr Anastasia Fialkov}, PhD Supervisor, Royal Society University Research Fellow, Institute of Astronomy, University of Cambridge. \textbf{Email:} anastasia.fialkov@gmail.com \\


\end{document}